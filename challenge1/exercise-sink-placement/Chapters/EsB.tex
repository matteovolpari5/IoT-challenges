In order to maximize the system's lifetime, we need to maximize the lifetime of the sensor node that consumes the most energy. The energy consumption for a transmission is given by:
\[
E_{cycle} = E_c \cdot b + E_{tx} \cdot b = E_c \cdot b + k \cdot d^2 \cdot b
\]
Since the energy for the circuitry, $E_c$, is constant, the energy consumption grows with the distance. In order to maximize the lifetime of the system, we need to minimize the maximum distance between the sink and the furthest node. This problem is equivalent to that of finding the smallest circle enclosing all point on the cartesian plane, representing the positions of nodes.\\
A very efficient algorithm that can be used to solve the smallest enclosing circle problem is Welzl's algorithm, which finds a solution in O(n) time. A Python implementation of the algorithm can be found below.

\begin{python}
import math
import random

# data structures
class Point:
    def __init__(self, x, y):
        self.x = x
        self.y = y

class Circle:
    def __init__(self, c, r):
        self.c = c
        self.r = r

# checks wheter point p is inside circle c 
def isInside(c, p):
    return math.dist([c.c.x, c.c.y], [p.x, p.y]) <= c.r

# checks if all points in ps are in c 
def isValidCircle(c, ps):
    return all(isInside(c, point) for point in ps)

# helper method to get a circle defined by 3 points
def getCircleCenter(bx, by, cx, cy):
    b = bx * bx + by * by
    c = cx * cx + cy * cy
    d = bx * cy - by * cx
    return Point((cy * b - by * c) / (2 * d), (bx * c - cx * b) / (2 * d))

# returns the circle passing for 2 points (a,b)
def circleFromTwo(a, b):
    c = Point((a.x + b.x) / 2.0, (a.y + b.y) / 2.0)
    return Circle(c, math.dist([a.x, a.y], [b.x, b.y]) / 2.0)

# returns the circle passing for 3 points (a,b,c)
def circleFromThree(a, b, c):
    i = getCircleCenter(b.x - a.x, b.y - a.y, c.x - a.x, c.y - a.y)
    i.x += a.x
    i.y += a.y
    return Circle(i, math.dist([i.x, i.y], [a.x, a.y]))

# minimum enclosing circle trivial cases
def minCircleTrivial(p):
    assert len(p) <= 3
    if not p:
        return Circle(Point(0, 0), 0)
    elif len(p) == 1:
        return Circle(p[0], 0)
    elif len(p) == 2:
        return circleFromTwo(p[0], p[1])

    for i in range(3):
        for j in range(i + 1, 3):
            c = circleFromTwo(p[i], p[j])
            if isValidCircle(c, p):
                return c
    return circleFromThree(p[0], p[1], p[2])

def welzl(p):
    pCopy = list(p)
    random.shuffle(pCopy)
    return welzlHelper(pCopy, [], len(pCopy))

def welzlHelper(p, r, n):
    if n == 0 or len(r) == 3:
        return minCircleTrivial(r[:])

    idx = random.randint(0, n - 1)
    pnt = p[idx]
    p[idx], p[n - 1] = p[n - 1], p[idx]

    d = welzlHelper(p, r, n - 1)

    if isInside(d, pnt):
        return d

    return welzlHelper(p, r + [pnt], n - 1)


# run algorithm on nodes positions
nodes_positions = [
    Point(1,2),
    Point(10,3),
    Point(4,8),
    Point(15,7),
    Point(6,1),
    Point(9,12),
    Point(14,4),
    Point(3,10),
    Point(7,7),
    Point(12,14)
]
mec = welzl(nodes_positions)

# print sink position
print("Sink position:", mec.c.x, mec.c.y)
print("Furthest node:", mec.r)
\end{python}

Running the algorithm, we get the following result:
\begin{verbatim}
Sink position: 6.871681415929204 7.65929203539823
Furthest node: 8.155012507169454
\end{verbatim}

Thus, the optimal position of the sink is approximately (6.87, 7.66) and the maximum distance from the sink to a node is of about 8.16 meters.\\
We can compute the lifetime of the system when the sink is in position (6.87, 7.66) as follows.

\[
E_{cycle, furthest} = E_c \cdot b + E_{tx, furthest} \cdot b = 0.233 mJ
\]

\[
n = \text{\# cycles} = E_b / E_{cycle, furthest} = 21.443 \text{ cycles}
\]

Under the assumption that each sensor transmits at the beginning of the ten minutes, it will last 21 cycles and die in the next one.




