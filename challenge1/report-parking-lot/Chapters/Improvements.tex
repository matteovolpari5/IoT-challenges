In order to improve the energy consumption of this IoT system, it is possible modify multiple parameters:
\begin{itemize}
	\item transmission power for ESP-NOW communication;
	\item delay used to make a measurement with HC-SR04;
	\item time spent with WiFi turned on;
	\item time spent in deep sleep.
\end{itemize}

\section{Improvements for our implementation}
In our implementation, we use the lowest transmission power, of 2 dBm, and the lowest time, suggested by the documentation, of 10 $\mu \text{s}$ to make measurement with HC-SR04 ultrasonic sensor. Finally, we perform the board setup and the measurement with WiFi turned off.\\
Thus, the only parameter which we can modify to improve the system's lifetime is the deep sleep time. We can tolerate a longer time to update the status of occupancy of the parking spot, but achieve a better lifetime of the node.\\
For instance, we can adopt a deep sleep time of two minutes and compute the new energy consumption and lifetime of the system.\\
The new energy consumption, with $T_{deep\_sleep} = 120 s$,  for the deep sleep phase is:
\begin{align*}
   	E_{deep\_sleep} &= P_{deep\_sleep} \cdot T_{deep\_sleep} = 7159.2\,\text{mJ} 
\end{align*}
Since all other values of energy consumptions are unchanged, we can compute the energy consumption of a cycle as:
\[
E_{cycle} = E_{idle} + E_{measurement} + E_{WiFi} + E_{transmission} + E_{deep\_sleep} = 7304,833\,\text{mJ} 
\]

The new sensor node's lifetime, measured in cycles, is:
\begin{align*}
	L_{cycles}&= E_{b}/E_{cycle} = 2545.301 \,\text{cycles} 
\end{align*}

The total time for a cycle is:
\begin{align*}
	T_{cycle} &= T_{idle} + T_{measurement} + T_{WiFi} + T_{deep\_sleep} = 120.208 \text{s}
\end{align*}

The lifetime, measured in time, is of:
\begin{align*}
	L_{time}&= L_{cycles} \cdot T_{cycle} = 305965.563 \text{s}
\end{align*}

The new lifetime, with $T_{deep\_sleep} = 120 s$, is equivalent to 3 days 12 hours 59 minutes and 25 seconds, which is more than 2 hours longer than the base case.

\section{Improvements upper bound}
In this section we prove that it is not possible to improve the lifetime of our system by a lot, by computing the lifetime if the deep sleep time tends toward infinity, i.e. the node always stays in deep sleep and isn't functional anymore.\\
The power consumption of the ESP32 in deep sleep mode is $P_{deep\_sleep} = 59.66\,\text{mW}$ and the battery life is $E_{b} = 18593\,\text{J}$, so the lifetime is:
\begin{align*}
	L_{time}&= E_{b} / P_{deep\_sleep} = 311649.346 \text{s}
\end{align*}

The lifetime of the system, when the deep sleep time tend towards infinity, is of 3 days 14 hours 34 minutes and 9 seconds, which is less than four hours bigger than the base case and less than two ours bigger than the improved version of the system.









