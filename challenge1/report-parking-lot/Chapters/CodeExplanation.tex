In this section, we will describe the main structure of the code written to meet the specification provided to us.
\section{Global variables}
\subsection{Auxiliary variables}
We have defined two global variables that are not strictly necessary for the correct operation of the system but are useful for future points of the challenge.
\begin{verbatim}
#define DEBUG 0
\end{verbatim}
Variable which when set to 1 is used in testing to print the messages arriving in broadcast to the board and the status of the messages sent. This variable also introduces a minimum delay of 10 ms to print the messages before going into deep sleep.\\
\begin{verbatim}
#define TIME_MEASUREMENT 1
\end{verbatim}
This variable when set to 1 is used to measure all the inter-times of the various phases of a system operation cycle. To function correctly, \textit{DEBUG} must be set to 0, otherwise the data collected is distorted by the delay introduced by \textit{DEBUG}.


\subsection{Pin declaration}
\begin{verbatim}
#define TRIG 4
#define ECHO 2
\end{verbatim}
Here we have defined 2 variables which identify the functional pins of the HC-SR04 component and which we have connected to pins 4 and 2 of the ESP32 board.

\subsection{Exercise constants}
\begin{verbatim}
#define TIME_TO_SLEEP 48
\end{verbatim}
Variable used to define the time for which the board will go into deep sleep. The value 48 is calculated using the formula: \textit{TIME\_TO\_SLEEP} = 93\%50+5 = 48 \\
where 93 is coming from the person code 10773593. \\
\begin{verbatim}
#define uS_TO_S_FACTOR 1000000
\end{verbatim}
Constant equal to $10^6$ used to convert from seconds to $\mu\text{S}$.\\
\begin{verbatim}
#define DISTANCE_LIMIT 50
\end{verbatim}
Threshold distance of 50 cm set by the exercise to consider the parking space occupied or not.


\section{Auxiliary functions}
We have used various auxiliary functions to make the code more readable and functional, and this section will look at them one by one. \\
\begin{verbatim}
void setupESP_NOW()
\end{verbatim}
This function is used to switch on the WiFi and do the setup.
In the setup phase, we set the transmission power to 2 dBm and called two other auxiliary functions that are used in the debugging phase (\textit{OnDataSent} and \textit{OnDataRecv}). \\
\begin{verbatim}
void OnDataSent(const uint8_t *mac_addr, esp_now_send_status_t status)
void OnDataRecv(const uint8_t * mac, const uint8_t *data, int len) 
\end{verbatim}
These are two callback functions which are used in debugging to print out whether a message has been sent correctly or whether there have been errors and all messages arriving on the board. \\
\begin{verbatim}
float getDistanceSensorMeasurement()
\end{verbatim}
This function is used to perform a single measurement via the ultrasonic sensor, after taking the data it converts it into cm and returns it.

\section{Setup function}
The setup function is the heart of the system; each time it wakes up, the board performs this function to complete a life cycle.\\
The first thing this function does is set the data rate for serial data transmission to write to the serial.
Then the modes of the two pins \textit{TRIG} and \textit{ECHO} are set to output and input respectively.
After this, a measurement is made via the ultrasonic sensor and only then is the \textit{setupESP\_NOW()} function called to activate WiFi.\\
When the WiFi is switched on and ready to be used, a message is sent to the MAC address of the sink, which in this case is set to the broadcast address 8C:AA:B5:84:FB:90.\\
Once the measurement is sent, the WiFi is switched off to reduce the energy consumption and the system is sent into deep sleep.
On awakening from deep sleep the card will restart from the beginning of the setup function and will then perform all the operations already seen in the same order.







