\section{CQ1}
\subsubsection{Question}
How many different Confirmable PUT requests obtained an unsuccessful response from the local CoAP server?

\subsubsection{Answer}
TODO 

\subsubsection{Explanation}
Domanda 1
Confirmable put request 
\begin{verbatim}
coap && coap.type == 0 && coap.code == 3
\end{verbatim}
45 frames
Response 
\begin{verbatim}
coap && (coap.code >= 128)
\end{verbatim}
228 frames 
Dovrei matcharli per token o message id (quale????), troppi.
In realtà posso filtrare anche ip src = ip dst.
Quindi pyshark.

\section{CQ2}
\subsubsection{Question}
How many CoAP resources in the coap.me public server received the same number of unique Confirmable and Non Confirmable GET requests?\\
Assuming a resource receives X different CONFIRMABLE requests and Y different NONCONFIRMABLE GET requests, how many resources have X=Y, with X>0?

\subsubsection{Answer}
TODO 

\subsubsection{Explanation}
Domanda 2
Get request confirmable a coap.me
\begin{verbatim}
coap.type == 0 && coap.code == 1 && ip.dst==134.102.218.18
\end{verbatim}
39 frames
Get non confirmable a coap.me
\begin{verbatim}
coap.type == 1 && coap.code == 1 && ip.dst==134.102.218.18
\end{verbatim}
31 frames
Dovrei vedere a quale risorsa fanno riferimento e poi confrontare.
Troppo, quindi pyshark.

\section{CQ3}
\subsubsection{Question}
How many different MQTT clients subscribe to the public broker HiveMQ using multi-level wildcards?

\subsubsection{Answer}
The number of clients who subscribe to the public broker HiveMQ using multi-level wildcards is 4.

\subsubsection{Explanation}
In order to find the IP address of the HiveMQ broker, we filter the response of the DNS server using the following Wireshark filter:
\begin{verbatim}
dns.qry.name == "broker.hivemq.com"
\end{verbatim}
All DNS responses return 3 addresses: 18.192.151.104, 35.158.34.213 and 35.158.43.69.

We use a second filter to find SUBSCRIBE messages, with message type 8, sent to HiveMQ broker, to one of the IP addresses found above, with a multi-level wildcard, ending with "\#": 
\begin{verbatim}
mqtt && mqtt.msgtype == 8 && 
(ip.dst == 18.192.151.104 || ip.dst == 35.158.34.213 
|| ip.dst == 35.158.43.69) && mqtt.topic contains "#"
\end{verbatim}
We find out that HiveMQ broker receives 6 messages of this type, all at the IP address 18.192.151.104.

\begin{figure}[H]
    \centering
    \includegraphics[width=\linewidth, height=0.85\textheight, keepaspectratio]{QC3_1.png}
    \caption{SUBSCRIBE messages to HiveMQ broker with "\#"}
\end{figure}

Since the question asks for the number of MQTT clients who subscribe, we need to identify the clients who sent these messages. For each message, we select the TCP stream, which identifies the client.\\

\begin{table}[H]
\centering 
\begin{tabular}{| c | c |}
	\hline 
	\rowcolor{bluepoli!40}
	\textbf{Message number} & \textbf{TCP stream}\T\B \\
	\hline 
	375 & 8 \T\B\\
	2442 & 15 \T\B\\
	3293 & 20 \T\B\\
	3303 & 15 \T\B\\
	3362 & 3 \T\B\\
	3693  & 15 \T\B\\
	\hline
\end{tabular}
\\[10pt]
\caption{TCP streams}
\label{table:tcp_streams}
\end{table}

Since there are 4 TCP streams, the 6 messages have been sent by 4 different client.	\\
We can also find the Client ID of these clients by finding the CONNECT message, of type 1, they sent to the broker. For the TCP stream 8, we can use the following filter:
\begin{verbatim}
mqtt && mqtt.msgtype == 1 && tcp.stream == 8
\end{verbatim}
The same filter with different TCP stream can be used for other clients.

\begin{table}[H]
\centering 
\begin{tabular}{| c | c |}
	\hline 
	\rowcolor{bluepoli!40}
	\textbf{TCP stream} & \textbf{Client ID}\T\B \\
	\hline 
	3 & cpoepjzkhibxgjiu \T\B\\
	8 & dzcxnwdqef \T\B\\
	15 & tukvxesuhe \T\B\\
	20 & fcthvjikxjul \T\B\\
	\hline
\end{tabular}
\\[10pt]
\caption{Client IDs table}
\label{table:client_ids_table}
\end{table}

\section{CQ4}
\subsubsection{Question}
How many different MQTT clients specify a Last Will Message to be directed to a topic having as first level "university"?

\subsubsection{Answer}
The number of clients who specify a Last Will Message to be directed to a topic having as first level "university" is 1.

\subsubsection{Explanation}
MQTT clients can specify a Last Will Message in the CONNECT message. In order to find the described messages, we filter CONNECT messages, of type 1, with a Last Will Topic:
\begin{verbatim}
mqtt && mqtt.msgtype == 1 && mqtt.willtopic
\end{verbatim}
We find four messages, but only one of them has a Last Will Topic having as first level "university".\\

\begin{figure}[H]
    \centering
    \includegraphics[width=\linewidth, height=0.85\textheight, keepaspectratio]{QC4_1.png}
    \caption{CONNECT messages specifying a Last Will Topic}
\end{figure}

We can find the result by enriching the filter and avoiding manually checking the topics, using the following filter:
\begin{verbatim}
mqtt && mqtt.msgtype == 1 && mqtt.willtopic matches "^university"
\end{verbatim}
Using this filter, we directly get the only message asked by CQ4.

\begin{figure}[H]
    \centering
    \includegraphics[width=\linewidth, height=0.85\textheight, keepaspectratio]{QC4_2.png}
    \caption{CONNECT messages specifying a Last Will Topic starting with "university"}
\end{figure}

\section{CQ5}
\subsubsection{Question}
How many MQTT subscribers receive a last will message derived from a subscription without a wildcard?

\subsubsection{Answer}
The number of subscribers who receive a Last Will Message derived from a subscription without a wildcard is 3.

\subsubsection{Explanation}
We start by identifying the possible Last Will Messages (LWM). To do so, we find all the CONNECT messages, with message type 1, that specify a LWM.
\begin{verbatim}
mqtt && mqtt.msgtype== 1 && mqtt.willmsg
\end{verbatim}

We find four messages.
\begin{figure}[H]
    \centering
    \includegraphics[width=\linewidth, height=0.85\textheight, keepaspectratio]{QC5_1.png}
    \caption{CONNECT messages specifying a LWM}
\end{figure}

These messages specify the Last Will Messages, which we don't report in the table, and Last Will Topics.
\begin{table}[H]
\centering 
\begin{tabular}{| c | c | c |}
	\hline 
	\rowcolor{bluepoli!40}
	\textbf{Message number} & \textbf{Destination} & \textbf{Last Will Topic}\T\B \\
	\hline 
	4 & ::1 & university/department12/room1/temperature \T\B\\
	196 & 5.196.78.28 & metaverse/room2/floor4 \T\B\\
	352 & 5.196.78.28 & hospital/facility3/area3 \T\B\\
	557 & 5.196.78.28 & metaverse/room2/room2 \T\B\\
	\hline
\end{tabular}
\\[10pt]
\caption{Last Will Topics}
\end{table}

Starting from the first Last Will Topic (LWT), we filter all PUBLISH messages with that topic and with same message as the LWM, found in the CONNECT message.
\begin{verbatim}
mqtt && mqtt.msgtype==3 && 
mqtt.topic == "university/department12/room1/temperature" && 
mqtt.msg contains 6572726f723a20612056495020436c69656e74206a7573742064696564
\end{verbatim}

We find four messages:
\begin{figure}[H]
    \centering
    \includegraphics[width=\linewidth, height=0.85\textheight, keepaspectratio]{QC5_2.png}
    \caption{LWM with topic "university/department12/room1/temperature"}
\end{figure}

By looking at the messages, we can see a TCP Reset message, representing an hard disconnection, followed by four Last Will Messages sent by the broker, on port 1883, to different client, on different ports and using different TCP streams.
\begin{figure}[H]
    \centering
    \includegraphics[width=\linewidth, height=0.85\textheight, keepaspectratio]{QC5_3.png}
    \caption{TCP Reset and Last Will Messages}
\end{figure}

The clients who receive the LWM are grouped in the following table.
\begin{table}[H]
\centering 
\begin{tabular}{| c | c | c |}
	\hline 
	\rowcolor{bluepoli!40}
	\textbf{Message number} & \textbf{Subscriber port} & \textbf{TCP stream}\T\B \\
	\hline 
	6560 & 39551 & 2 \T\B\\
	6562 & 53557 & 6 \T\B\\
	6564 & 51743 & 10 \T\B\\
	6566 & 41789 & 14 \T\B\\
	\hline
\end{tabular}
\\[10pt]
\caption{Last Will Topics}
\end{table}

In order to answer to QC5, we need to find which of these clients subscribed to the Last Will Topic without a wildcard. We can do so by filtering the SUBSCRIBE messages, with message type 8, with the correct TCP stream. For each of them we also find the CONNECT message and Client ID.
\begin{verbatim}
mqtt && mqtt.msgtype == 8 && mqtt.topic matches "^university" && 
(tcp.stream == 2 || tcp.stream == 6 || tcp.stream == 10 || tcp.stream == 14)
\end{verbatim}

\begin{figure}[H]
    \centering
    \includegraphics[width=\linewidth, height=0.85\textheight, keepaspectratio]{QC5_4.png}
    \caption{TCP Reset and Last Will Messages}
\end{figure}

We can see that only three out of four clients subscribed to the Last Will Topic without a wildcard.

\begin{table}[H]
\centering 
\begin{tabular}{| c | c | c |}
	\hline 
	\rowcolor{bluepoli!40}
	\textbf{TCP stream} & \textbf{Client ID} & \textbf{Specified topic}\T\B \\
	\hline 
	2 & auyvhrhdudnm & university/department12/room1/temperature \T\B\\
	6 & ntpiopsqc & university/department12/room1/temperature \T\B\\
	10 & zmjnxudohrkaegmh & university/\# \T\B\\
	14 & mjdocmjxt & university/department12/room1/temperature \T\B\\
	\hline
\end{tabular}
\\[10pt]
\caption{Specified topics}
\end{table}

For what concerns the other three Last Will Topics, we filter all PUBLISH messages, with message type 3, from the broker, with IP 5.196.78.28.
\begin{verbatim}
mqtt && ip.src == 5.196.78.28 && mqtt.msgtype == 3 
\end{verbatim}
We don't find any result, which means that the broker doesn't publish any message, nor LWM.\\
We conclude that three clients receive a LWM from a subscription without a wildcard, all of them from the first topic.


TODO 
Però non ha retain true!
\begin{verbatim}
mqtt && mqtt.msgtype==3 && mqtt.topic == "university/department12/room1/temperature" && mqtt.msg contains 
6572726f723a20612056495020436c69656e74206a7573742064696564 && mqtt.retain == 1
\end{verbatim}
Nessun risultato
Devono avere il retain? Se sì, allora non sono last will, altrimenti???
Non penso. Il retain dovrebbe essere sul messaggio di connect, non su quello di last will!

\section{CQ6}
\subsubsection{Question}
How many MQTT publish messages directed to the public broker mosquitto are sent with the retain option and use QoS “At most once”?

\subsubsection{Answer}
TODO 

\subsubsection{Explanation}

Domanda 6:

utilizzando il filtro: dns.qry.name == "test.mosquitto.org" trovo l'indirizzo ip collegato al dominio richiesto che poi mi servirà per filtrare i pacchetti, l'IP è:  5.196.78.28.
quindi filtro i pacchetti con: mqtt.msgtype == 3 and mqtt.qos == 0 and mqtt.retain == 1 and ip.dst == 5.196.78.28 e trovo tutti quelli che rispettano la richiesta: 208 pacchetti.

\section{CQ7}
\subsubsection{Question}
How many MQTT-SN messages on port 1885 are sent by the clients to a broker in the local machine?

\subsubsection{Answer}
TODO 

\subsubsection{Explanation}

Domanda 7: 
Il protocollo MQTT-SN non è riconosciuto nativamente da Wireshark ma sappiamo da teoria che è un protocollo udp. Filtrando quindi con udp.port == 1885 non trovo nessun pacchetto e quindi non sono stati inviati pacchetti sulla porta 1885.


References:
\begin{verbatim}
https://docs.oasis-open.org/mqtt/mqtt/v5.0/os/mqtt-v5.0-os.html#_Toc3901022
\end{verbatim}






















