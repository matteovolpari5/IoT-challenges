\section{CQ1}
\subsubsection{Question}
How many different Confirmable PUT requests obtained an unsuccessful response from the local CoAP server?

\subsubsection{Answer}
TODO 

\subsubsection{Explanation}
Domanda 1
Confirmable put request 
\begin{verbatim}
coap && coap.type == 0 && coap.code == 3
\end{verbatim}
45 frames
Response 
\begin{verbatim}
coap && (coap.code >= 128)
\end{verbatim}
228 frames 
Dovrei matcharli per token o message id (quale????), troppi.
In realtà posso filtrare anche ip src = ip dst.
Quindi pyshark.

\section{CQ2}
\subsubsection{Question}
How many CoAP resources in the coap.me public server received the same number of unique Confirmable and Non Confirmable GET requests?\\
Assuming a resource receives X different CONFIRMABLE requests and Y different NONCONFIRMABLE GET requests, how many resources have X=Y, with X>0?

\subsubsection{Answer}
TODO 

\subsubsection{Explanation}
Domanda 2
Get request confirmable a coap.me
\begin{verbatim}
coap.type == 0 && coap.code == 1 && ip.dst==134.102.218.18
\end{verbatim}
39 frames
Get non confirmable a coap.me
\begin{verbatim}
coap.type == 1 && coap.code == 1 && ip.dst==134.102.218.18
\end{verbatim}
31 frames
Dovrei vedere a quale risorsa fanno riferimento e poi confrontare.
Troppo, quindi pyshark.

\section{CQ3}
\subsubsection{Question}
How many different MQTT clients subscribe to the public broker HiveMQ using multi-level wildcards?

\subsubsection{Answer}
The number of clients who subscribe to the public broker HiveMQ using multi-level wildcards is 4.

\subsubsection{Explanation}
In order to find the IP address of the HiveMQ broker, we filter the response of the DNS server using the following Wireshark filter:
\begin{verbatim}
dns.qry.name == "broker.hivemq.com"
\end{verbatim}
All DNS responses return 3 addresses: 18.192.151.104, 35.158.34.213 and 35.158.43.69.

We use a second filter to find SUBSCRIBE messages, with message type 8, sent to HiveMQ broker, to one of the IP addresses found above, with a multi-level wildcard, ending with "\#": 
\begin{verbatim}
mqtt && mqtt.msgtype == 8 && 
(ip.dst == 18.192.151.104 || ip.dst == 35.158.34.213 
|| ip.dst == 35.158.43.69) && mqtt.topic contains "#"
\end{verbatim}
We find out that HiveMQ broker receives 6 messages of this type, all at the IP address 18.192.151.104.

Since the question asks for the number of MQTT clients who subscribe, we need to identify the clients who sent these messages. For each message, we select the TCP stream, which identifies the client.\\

\begin{table}[H]
\centering 
\begin{tabular}{| c | c |}
	\hline 
	\rowcolor{bluepoli!40}
	\textbf{Message number} & \textbf{TCP stream}\T\B \\
	\hline 
	375 & 8 \T\B\\
	2442 & 15 \T\B\\
	3293 & 20 \T\B\\
	3303 & 15 \T\B\\
	3362 & 3 \T\B\\
	3693  & 15 \T\B\\
	\hline
\end{tabular}
\\[10pt]
\caption{TCP streams}
\label{table:tcp_streams}
\end{table}

Since there are 4 TCP streams, the 6 messages have been sent by 4 different client.	\\
We can also find the Client ID of these clients by finding the CONNECT message, of type 1, they sent to the broker. For the TCP stream 8, we can use the following filter:
\begin{verbatim}
mqtt && mqtt.msgtype == 1 && tcp.stream == 3 
\end{verbatim}
The same filter with different TCP stream can be used for other clients.

\begin{table}[H]
\centering 
\begin{tabular}{| c | c |}
	\hline 
	\rowcolor{bluepoli!40}
	\textbf{TCP stream} & \textbf{Client ID}\T\B \\
	\hline 
	3 & cpoepjzkhibxgjiu \T\B\\
	8 & dzcxnwdqef \T\B\\
	15 & tukvxesuhe \T\B\\
	20 & fcthvjikxjul \T\B\\
	\hline
\end{tabular}
\\[10pt]
\caption{Client IDs table}
\label{table:client_ids_table}
\end{table}

\section{CQ4}
\subsubsection{Question}
How many different MQTT clients specify a Last Will Message to be directed to a topic having as first level "university"?

\subsubsection{Answer}
The number of clients who specify a Last Will Message to be directed to a topic having as first level "university" is 1.

\subsubsection{Explanation}
MQTT clients can specify a Last Will Message in the CONNECT message. In order to find the described messages, we filter CONNECT messages, of type 1, with a Last Will Topic:
\begin{verbatim}
mqtt && mqtt.msgtype == 1 && mqtt.willtopic
\end{verbatim}
We find four messages, but only one of them has a Last Will Topic having as first level "university".\\
We can find the result by enriching the filter and avoiding manually checking the topics, using the following filter:
\begin{verbatim}
mqtt && mqtt.msgtype == 1 && mqtt.willtopic matches "^university"
\end{verbatim}
Using this filter, we directly get the only message asked by CQ4.

\section{CQ5}
\subsubsection{Question}
How many MQTT subscribers receive a last will message derived from a subscription without a wildcard?

\subsubsection{Answer}
TODO 

\subsubsection{Explanation}

Domanda 5:

\begin{verbatim}
mqtt && mqtt.msgtype== 1 && mqtt.willmsg
\end{verbatim}
Ottengo 4 fram di tipo CONNECT che specificano un last will message e last will topic.

4	0.000117188	::1	::1	MQTT	176	Connect Command
196	2.116585177	10.0.2.15	5.196.78.28	MQTT	126	Connect Command
352	5.034840089	10.0.2.15	5.196.78.28	MQTT	123	Connect Command
557	7.043177949	10.0.2.15	5.196.78.28	MQTT	120	Connect Command

1
Will Topic: university/department12/room1/temperature
Will Message Length: 29
Will Message: 6572726f723a20612056495020436c69656e74206a7573742064696564
2
Will Topic: metaverse/room2/floor4
Will Message Length: 15
Will Message: 6572726f723a20706575697664716c
3
Will Topic: hospital/facility3/area3
Will Message Length: 15
Will Message: 6572726f723a20786c7a6171707464
4
Will Topic: metaverse/room2/room2
Will Message Length: 15
Will Message: 6572726f723a207a6a7a7772636470

Per il primo:
Messaggi pubblicati su quel topic:
\begin{verbatim}
mqtt && mqtt.msgtype==3 && mqtt.topic == "university/department12/room1/temperature" 
\end{verbatim}
4 risultati
Messaggi pubblicati su quel topic con messaggio uguale a last will message:
\begin{verbatim}
mqtt && mqtt.msgtype==3 && mqtt.topic == "university/department12/room1/temperature" && mqtt.msg contains 
\end{verbatim}
6572726f723a20612056495020436c69656e74206a7573742064696564
Stessi 4 risultati
Però non ha retain true!
\begin{verbatim}
mqtt && mqtt.msgtype==3 && mqtt.topic == "university/department12/room1/temperature" && mqtt.msg contains 
6572726f723a20612056495020436c69656e74206a7573742064696564 && mqtt.retain == 1
\end{verbatim}
Nessun risultato

Devono avere il retain? Se sì, allora non sono last will, altrimenti???
Non penso. Il retain dovrebbe essere sul messaggio di connect, non su quello di last will!

Inoltre sono i pacchetti 6560, 6562, 6564, 6566. A 6559 c'è un reset ??.
Si tratta di questi pacchetti 
6559	146.691800286	::1	::1	TCP	88	38083 → 1883 [RST, ACK] Seq=9316 Ack=85 Win=65536 Len=0 TSval=2654936537 TSecr=2654934931
6560	146.692096889	::1	::1	MQTT	162	Publish Message [university/department12/room1/temperature]
6561	146.692172650	::1	::1	TCP	88	39551 → 1883 [ACK] Seq=77 Ack=86 Win=65536 Len=0 TSval=2654936537 TSecr=2654936537
6562	146.692187516	::1	::1	MQTT	164	Publish Message (id=1) [university/department12/room1/temperature]
6563	146.692189976	::1	::1	TCP	88	53557 → 1883 [ACK] Seq=94 Ack=86 Win=65536 Len=0 TSval=2654936537 TSecr=2654936537
6564	146.692199911	::1	::1	MQTT	165	Publish Message (id=12) [university/department12/room1/temperature]
6565	146.692202117	::1	::1	TCP	88	51743 → 1883 [ACK] Seq=573 Ack=14152 Win=64896 Len=0 TSval=2654936537 TSecr=2654936537
6566	146.692209983	::1	::1	MQTT	164	Publish Message (id=1) [university/department12/room1/temperature]
6560, 6562, 6564 e 6466 sono quelli con contenuto pari al last will message sul last will topic.
Per questi pacchetti, ho notato che la source port è sempre 1883, mentre la destination port cambia.
Infine il tcp stream cambia.

Devo controllare se si sono iscritti con wildcard, seguo il tcp stream e identifico le richieste di SUBSCRIBE.

6560
Destination Port: 39551
[Stream index: 2]
\begin{verbatim}
mqtt && mqtt.msgtype == 8 && tcp.stream == 2
\end{verbatim}
Non usa wildcard
\begin{verbatim}
mqtt && mqtt.msgtype == 1 && tcp.stream == 2
\end{verbatim}
Client ID: auyvhrhdudnm

6562
Destination Port: 53557
[Stream index: 6]
\begin{verbatim}
mqtt && mqtt.msgtype == 8 && tcp.stream == 6
\end{verbatim}
Non usa wildcard
\begin{verbatim}
mqtt && mqtt.msgtype == 1 && tcp.stream == 6
\end{verbatim}
Client ID: pcdwkgeslfh

6564
Destination Port: 51743
[Stream index: 10]
\begin{verbatim}
mqtt && mqtt.msgtype == 8 && tcp.stream == 10
\end{verbatim}
Molte subscription: una sola che va bene per il nostro topic, con wildcard 
\begin{verbatim}
1136	10.102492445	::1	::1	MQTT	108	Subscribe Request (id=6) [university/#]
\end{verbatim}

Destination Port: 41789
[Stream index: 14]
\begin{verbatim}
mqtt && mqtt.msgtype == 8 && tcp.stream == 14
\end{verbatim}
Non usa wildcard
\begin{verbatim}
mqtt && mqtt.msgtype == 1 && tcp.stream == 14
\end{verbatim}
Client ID: mjdocmjxt

RST è un reset, indica disconnessione brusca.
Successivamente il broker, sempre porta 1883, manda messaggi tutti con last will topic e last will message in loopback, sempre a porte diverse.
Secondo me sono dei last will messages.

Per quelli da 2 a 4:
5.196.78.28 è src solamente per PUBACK, PUBREC, CONNACK.
Filtro che estrae i messaggi pubblicati:
\begin{verbatim}
mqtt && ip.src == 5.196.78.28 && mqtt.msgtype == 3 
\end{verbatim}
non ritorna risultati
Se non ha pubblicato nulla, non può aver mandato dei last will message.

Conclusione: ci sono 4 CONNECT con last will, una di queste, quella in locale, ha dei messaggi di last will. 
Il messaggio viene inviato dopo un reset dal broker a 4 subscriber, vedendo la loro subscription, solo 3 su 4 non hanno usato la wildcard.

References:
\begin{verbatim}
https://docs.oasis-open.org/mqtt/mqtt/v5.0/os/mqtt-v5.0-os.html#_Toc3901022
\end{verbatim}

\section{CQ6}
\subsubsection{Question}
How many MQTT publish messages directed to the public broker mosquitto are sent with the retain option and use QoS “At most once”?

\subsubsection{Answer}
TODO 

\subsubsection{Explanation}

Domanda 6:

utilizzando il filtro: dns.qry.name == "test.mosquitto.org" trovo l'indirizzo ip collegato al dominio richiesto che poi mi servirà per filtrare i pacchetti, l'IP è:  5.196.78.28.
quindi filtro i pacchetti con: mqtt.msgtype == 3 and mqtt.qos == 0 and mqtt.retain == 1 and ip.dst == 5.196.78.28 e trovo tutti quelli che rispettano la richiesta: 208 pacchetti.

\section{CQ7}
\subsubsection{Question}
How many MQTT-SN messages on port 1885 are sent by the clients to a broker in the local machine?

\subsubsection{Answer}
TODO 

\subsubsection{Explanation}

Domanda 7: 
Il protocollo MQTT-SN non è riconosciuto nativamente da Wireshark ma sappiamo da teoria che è un protocollo udp. Filtrando quindi con udp.port == 1885 non trovo nessun pacchetto e quindi non sono stati inviati pacchetti sulla porta 1885.

























