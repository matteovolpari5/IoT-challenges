\documentclass{Configuration_Files/PoliMi3i_thesis}

% CONFIGURATIONS
\usepackage{parskip} % For paragraph layout
\usepackage{setspace} % For using single or double spacing
\usepackage{emptypage} % To insert empty pages
\usepackage{multicol} % To write in multiple columns (executive summary)
\setlength\columnsep{15pt} % Column separation in executive summary
\setlength\parindent{0pt} % Indentation
\raggedbottom  

% PACKAGES FOR TITLES
\usepackage{titlesec}
% \titlespacing{\section}{left spacing}{before spacing}{after spacing}
\titlespacing{\section}{0pt}{3.3ex}{2ex}
\titlespacing{\subsection}{0pt}{3.3ex}{1.65ex}
\titlespacing{\subsubsection}{0pt}{3.3ex}{1ex}
\usepackage{color}

% PACKAGES FOR LANGUAGE AND FONT
\usepackage[english]{babel} % The document is in English  
\usepackage[utf8]{inputenc} % UTF8 encoding
\usepackage[T1]{fontenc} % Font encoding
\usepackage[11pt]{moresize} % Big fonts

% PACKAGES FOR IMAGES 
\usepackage{graphicx}
\usepackage{transparent} % Enables transparent images
\usepackage{eso-pic} % For the background picture on the title page
\usepackage{subfig} % Numbered and caption subfigures using \subfloat.
\usepackage{tikz} % A package for high-quality hand-made figures.
\usetikzlibrary{}
\graphicspath{{./Images/}} % Directory of the images
\usepackage{caption} % Coloured captions
\usepackage{xcolor} % Coloured captions
\usepackage{amsthm,thmtools,xcolor} % Coloured "Theorem"
\usepackage{float}

% STANDARD MATH PACKAGES
\usepackage{amsmath}
\usepackage{amsthm}
\usepackage{amssymb}
\usepackage{amsfonts}
\usepackage{bm}
\usepackage[overload]{empheq} % For braced-style systems of equations.
\usepackage{fix-cm} % To override original LaTeX restrictions on sizes

% PACKAGES FOR TABLES
\usepackage{tabularx}
\usepackage{longtable} % Tables that can span several pages
\usepackage{colortbl}

% PACKAGES FOR ALGORITHMS (PSEUDO-CODE)
\usepackage{algorithm}
\usepackage{algorithmic}

% PACKAGES FOR REFERENCES & BIBLIOGRAPHY
\usepackage[
	colorlinks=true,
	linkcolor=black,
	anchorcolor=black,
	citecolor=black,
	filecolor=black,
	menucolor=black,
	runcolor=black,
	urlcolor=black
]{hyperref} % Adds clickable links at references
\usepackage{cleveref}
\usepackage[square, numbers, sort&compress]{natbib} % Square brackets, citing references with numbers, citations sorted by appearance in the text and compressed
\bibliographystyle{abbrvnat} % You may use a different style adapted to your field

% OTHER PACKAGES
\usepackage{pdfpages} % To include a pdf file
\usepackage{afterpage}
\usepackage{lipsum} % Dummy package
\usepackage{fancyhdr} % For the headers
\usepackage{enumitem}
\fancyhf{}
\usepackage{listingsutf8}
\usepackage{xcolor}

% Input of configuration file. Do not change config.tex file unless you really know what you are doing. 
% Define blue color typical of polimi
\definecolor{bluepoli}{cmyk}{0.4,0.1,0,0.4}

% Custom theorem environments
\declaretheoremstyle[
  headfont=\color{bluepoli}\normalfont\bfseries,
  bodyfont=\color{black}\normalfont\itshape,
]{colored}

% Set-up caption colors
\captionsetup[figure]{labelfont={color=bluepoli}} % Set colour of the captions
\captionsetup[table]{labelfont={color=bluepoli}} % Set colour of the captions
\captionsetup[algorithm]{labelfont={color=bluepoli}} % Set colour of the captions

\theoremstyle{colored}
\newtheorem{theorem}{Theorem}[chapter]
\newtheorem{proposition}{Proposition}[chapter]

% Enhances the features of the standard "table" and "tabular" environments.
\newcommand\T{\rule{0pt}{2.6ex}}
\newcommand\B{\rule[-1.2ex]{0pt}{0pt}}

% Pseudo-code algorithm descriptions.
\newcounter{algsubstate}
\renewcommand{\thealgsubstate}{\alph{algsubstate}}
\newenvironment{algsubstates}
  {\setcounter{algsubstate}{0}%
   \renewcommand{\STATE}{%
     \stepcounter{algsubstate}%
     \Statex {\small\thealgsubstate:}\space}}
  {}

% New font size
\newcommand\numfontsize{\@setfontsize\Huge{200}{60}}

% Title format: chapter
\titleformat{\chapter}[hang]{
\fontsize{50}{20}\selectfont\bfseries\filright}{\textcolor{bluepoli} \thechapter\hsp\hspace{2mm}\textcolor{bluepoli}{|   }\hsp}{0pt}{\huge\bfseries \textcolor{bluepoli}
}

% Title format: section
\titleformat{\section}
{\color{bluepoli}\normalfont\Large\bfseries}
{\color{bluepoli}\thesection.}{1em}{}

% Title format: subsection
\titleformat{\subsection}
{\color{bluepoli}\normalfont\large\bfseries}
{\color{bluepoli}\thesubsection.}{1em}{}

% Title format: subsubsection
\titleformat{\subsubsection}
{\color{bluepoli}\normalfont\large\bfseries}
{\color{bluepoli}\thesubsubsection.}{1em}{}

% Shortening for setting no horizontal-spacing
\newcommand{\hsp}{\hspace{0pt}}

\makeatletter
% Renewcommand: cleardoublepage including the background pic
\let\cleardoublepage\clearpage

%For correctly numbering algorithms
\numberwithin{algorithm}{chapter}

%----------------------------------------------------------------------------
%	BEGIN OF YOUR DOCUMENT
%----------------------------------------------------------------------------

\begin{document}

\fancypagestyle{plain}{%
\fancyhf{} % Clear all header and footer fields
\fancyhead[RO,RE]{\thepage} %RO=right odd, RE=right even
\renewcommand{\headrulewidth}{0pt}
\renewcommand{\footrulewidth}{0pt}}

%----------------------------------------------------------------------------
%	TITLE PAGE
%----------------------------------------------------------------------------

\pagestyle{empty} % No page numbers
\frontmatter % Use roman page numbering style (i, ii, iii, iv...) for the preamble pages

\puttitle{
	title={IoT Challenge \#2,\\Packet Sniffing},% Title of the thesis
	course={Internet of Things},
	name1={Kevin Ziroldi - 10764177},
	name2={Matteo Volpari - 10773593},
	advisors= {Alessandro Redondi, Fabio Palmese}, % Supervisor name
	academicyear={2024-2025},  % Academic Year
	version={1.0}, 
	releasedate={6-4-2025}
} 
          
%----------------------------------------------------------------------------
%	PREAMBLE PAGES: ABSTRACT (inglese e italiano), EXECUTIVE SUMMARY
%----------------------------------------------------------------------------
\startpreamble
\setcounter{page}{1} % Set page counter to 1

%----------------------------------------------------------------------------
%	LIST OF CONTENTS/FIGURES/TABLES/SYMBOLS
%----------------------------------------------------------------------------

% TABLE OF CONTENTS
\thispagestyle{empty}
\tableofcontents % Table of contents 
\thispagestyle{empty}
\cleardoublepage

%-------------------------------------------------------------------------
%	THESIS MAIN TEXT
%-------------------------------------------------------------------------
% In the main text of your thesis you can write the chapters in two different ways:
%
%(1) As presented in this template you can write:
%    \chapter{Title of the chapter}
%    *body of the chapter*
%
%(2) You can write your chapter in a separated .tex file and then include it in the main file with the following command:
%    \chapter{Title of the chapter}
%    \input{chapter_file.tex}
%
% Especially for long thesis, we recommend you the second option.

% The \label{...}% enables to remove the small indentation that is generated, always leave the % symbol.

\addtocontents{toc}{\vspace{2em}} % Add a gap in the Contents, for aesthetics
\mainmatter % Begin numeric (1,2,3...) page numbering

% --------------------------------------------------------------------------
% NUMBERED CHAPTERS % Regular chapters following
% --------------------------------------------------------------------------

\chapter{Packet sniffing PCAP file}
\label{ch:chapter_one}%
Domanda 1
Confirmable put request 
coap && coap.type == 0 && coap.code == 3
45 frames
Response 
coap && (coap.code >= 128)
228 frames 
Dovrei matcharli per token/message id, troppi.
Quindi pyshark.

Domanda 2
Get request confirmable a coap.me
coap.type == 0 && coap.code == 1 && ip.dst==134.102.218.18
39 frames
Get non confirmable a coap.me
coap.type == 1 && coap.code == 1 && ip.dst==134.102.218.18
31 frames
Dovrei vedere a quale risorsa fanno riferimento e poi confrontare.
Troppo, quindi pyshark.

Domanda 3
dns.qry.name == "broker.hivemq.com"
::1 dovrebbe essere il loopback in ipv6, tipo 127.0.0.1 per ipv4

Filtro wireshark:
mqtt && mqtt.msgtype == 8 && ip.dst == 18.192.151.104 && mqtt.topic contains "#"

Ottieni 6 pacchetti.
Per ognuno prendi tcp stream:
N 375 - TCP stream = 8
N 2442 - TCP stream 15
N 3293 - TCP stream 20
N 3303 - TCP stream 15
N 3362 - TCP stream 3 
N 3693 - TCP stream 15 

Per ogni stream prendo il client id 
mqtt && mqtt.msgtype == 1 && tcp.stream == 3 
uguale per tutti gli stream, cambiando valore, ottengo client id 
Stream 3 - Client ID: cpoepjzkhibxgjiu
Stream 8 - Client ID: dzcxnwdqef
Stream 15 - Client ID: tukvxesuhe
Stream 20 - Client ID: fcthvjikxjul
Tutti diversi, 4 client.

Domanda 4
Filtro wireshark:
mqtt && mqtt.msgtype == 1 && mqtt.willtopic matches "^university"

Oppure:
mqtt && mqtt.msgtype == 1 && mqtt.willtopic
Ne escono 4 e li controllo, solo uno inizia per university 

Domanda 5:

mqtt && mqtt.msgtype== 1 && mqtt.willmsg
Ottengo 4 fram di tipo CONNECT che specificano un last will message e last will topic.

4	0.000117188	::1	::1	MQTT	176	Connect Command
196	2.116585177	10.0.2.15	5.196.78.28	MQTT	126	Connect Command
352	5.034840089	10.0.2.15	5.196.78.28	MQTT	123	Connect Command
557	7.043177949	10.0.2.15	5.196.78.28	MQTT	120	Connect Command

1
Will Topic: university/department12/room1/temperature
Will Message Length: 29
Will Message: 6572726f723a20612056495020436c69656e74206a7573742064696564
2
Will Topic: metaverse/room2/floor4
Will Message Length: 15
Will Message: 6572726f723a20706575697664716c
3
Will Topic: hospital/facility3/area3
Will Message Length: 15
Will Message: 6572726f723a20786c7a6171707464
4
Will Topic: metaverse/room2/room2
Will Message Length: 15
Will Message: 6572726f723a207a6a7a7772636470

Per il primo:
Messaggi pubblicati su quel topic:
mqtt && mqtt.msgtype==3 && mqtt.topic == "university/department12/room1/temperature" 
4 risultati
Messaggi pubblicati su quel topic con messaggio uguale a last will message:
mqtt && mqtt.msgtype==3 && mqtt.topic == "university/department12/room1/temperature" && mqtt.msg contains 6572726f723a20612056495020436c69656e74206a7573742064696564
Stessi 4 risultati
Però non ha retain true!
mqtt && mqtt.msgtype==3 && mqtt.topic == "university/department12/room1/temperature" && mqtt.msg contains 6572726f723a20612056495020436c69656e74206a7573742064696564 && mqtt.retain == 1
Nessun risultato

Devono avere il retain? Se sì, allora non sono last will, altrimenti???
Non penso. Il retain dovrebbe essere sul messaggio di connect, non su quello di last will!

Inoltre sono i pacchetti 6560, 6562, 6564, 6566. A 6559 c'è un reset ??.
Si tratta di questi pacchetti 
6559	146.691800286	::1	::1	TCP	88	38083 → 1883 [RST, ACK] Seq=9316 Ack=85 Win=65536 Len=0 TSval=2654936537 TSecr=2654934931
6560	146.692096889	::1	::1	MQTT	162	Publish Message [university/department12/room1/temperature]
6561	146.692172650	::1	::1	TCP	88	39551 → 1883 [ACK] Seq=77 Ack=86 Win=65536 Len=0 TSval=2654936537 TSecr=2654936537
6562	146.692187516	::1	::1	MQTT	164	Publish Message (id=1) [university/department12/room1/temperature]
6563	146.692189976	::1	::1	TCP	88	53557 → 1883 [ACK] Seq=94 Ack=86 Win=65536 Len=0 TSval=2654936537 TSecr=2654936537
6564	146.692199911	::1	::1	MQTT	165	Publish Message (id=12) [university/department12/room1/temperature]
6565	146.692202117	::1	::1	TCP	88	51743 → 1883 [ACK] Seq=573 Ack=14152 Win=64896 Len=0 TSval=2654936537 TSecr=2654936537
6566	146.692209983	::1	::1	MQTT	164	Publish Message (id=1) [university/department12/room1/temperature]
6560, 6562, 6564 e 6466 sono quelli con contenuto pari al last will message sul last will topic.
Per questi pacchetti, ho notato che la source port è sempre 1883, mentre la destination port cambia.
Infine il tcp stream cambia.

Devo controllare se si sono iscritti con wildcard, seguo il tcp stream e identifico le richieste di SUBSCRIBE.

6560
Destination Port: 39551
[Stream index: 2]
mqtt && mqtt.msgtype == 8 && tcp.stream == 2
Non usa wildcard
mqtt && mqtt.msgtype == 1 && tcp.stream == 2
Client ID: auyvhrhdudnm

6562
Destination Port: 53557
[Stream index: 6]
mqtt && mqtt.msgtype == 8 && tcp.stream == 6
Non usa wildcard
mqtt && mqtt.msgtype == 1 && tcp.stream == 6
Client ID: pcdwkgeslfh

6564
Destination Port: 51743
[Stream index: 10]
mqtt && mqtt.msgtype == 8 && tcp.stream == 10
Molte subscription: una sola che va bene per il nostro topic, con wildcard 
1136	10.102492445	::1	::1	MQTT	108	Subscribe Request (id=6) [university/#]

Destination Port: 41789
[Stream index: 14]
mqtt && mqtt.msgtype == 8 && tcp.stream == 14
Non usa wildcard
mqtt && mqtt.msgtype == 1 && tcp.stream == 14
Client ID: mjdocmjxt

RST è un reset, indica disconnessione brusca.
Successivamente il broker, sempre porta 1883, manda messaggi tutti con last will topic e last will message in loopback, sempre a porte diverse.
Secondo me sono dei last will messages.

Per quelli da 2 a 4:
5.196.78.28 è src solamente per PUBACK, PUBREC, CONNACK.
Filtro che estrae i messaggi pubblicati:
mqtt && ip.src == 5.196.78.28 && mqtt.msgtype == 3 
non ritorna risultati
Se non ha pubblicato nulla, non può aver mandato dei last will message.

Conclusione: ci sono 4 CONNECT con last will, una di queste, quella in locale, ha dei messaggi di last will. 
Il messaggio viene inviato dopo un reset dal broker a 4 subscriber, vedendo la loro subscription, solo 3 su 4 non hanno usato la wildcard.

References:
https://docs.oasis-open.org/mqtt/mqtt/v5.0/os/mqtt-v5.0-os.html#_Toc3901022



















%-------------------------------------------------------------------------
%	BIBLIOGRAPHY
%-------------------------------------------------------------------------

%\addtocontents{toc}{\vspace{2em}} % Add a gap in the Contents, for aesthetics
%\bibliography{Thesis_bibliography} % The references information are stored in the file named "Thesis_bibliography.bib"

%-------------------------------------------------------------------------
%	APPENDICES
%-------------------------------------------------------------------------

% LIST OF FIGURES
%\listoffigures

% LIST OF TABLES
%\listoftables


\end{document}

